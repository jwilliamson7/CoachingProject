\documentclass[USenglish]{article}

\usepackage[utf8]{inputenc}
\usepackage{xparse}
\usepackage[big,online]{dgruyter}
\usepackage{lmodern}
\usepackage{microtype}
\usepackage[authoryear,round,sort&compress]{natbib}
\usepackage{graphicx}
\usepackage{booktabs}
\usepackage{multirow}
\usepackage{array}
\usepackage{amsmath}

\theoremstyle{dgthm}
\newtheorem{theorem}{Theorem}
\newtheorem{corollary}{Corollary}
\newtheorem{proposition}{Proposition}
\newtheorem{lemma}{Lemma}

\theoremstyle{dgdef}
\newtheorem{definition}{Definition}
\newtheorem{example}{Example}
\newtheorem{remark}{Remark}

\begin{document}

%%%--------------------------------------------%%%
\articletype{Research Article}
\received{Month DD, YYYY}
\revised{Month DD, YYYY}
\accepted{Month DD, YYYY}
\journalname{Journal of Quantitative Analysis in Sports}
\journalyear{YYYY}
\journalvolume{XX}
\journalissue{X}
\startpage{1}
\aop
\DOI{10.1515/jqas-YYYY-XXXX}
%%%--------------------------------------------%%%

\title{Predicting NFL Head Coach Tenure Using Ordinal Classification}
\runningtitle{Predicting NFL Head Coach Tenure}

% Author information removed for blind peer review
%\author*[1]{Jon Williamson}
%\runningauthor{J.~Williamson}
%\affil[1]{\protect\raggedright
%Des Moines, Iowa, e-mail: jon@williamsonconsultinggroup.com}

\abstract{What if NFL teams could predict which head coach candidates would be successful? This project predicts the tenure classification of NFL head coach hires using statistics available at the time the hire was made. Using 150 engineered features spanning coach experience, historical team performance during prior coordinator and head coaching roles, and hiring team context, we implement ordinal classification via the Frank-Hall binary decomposition method with XGBoost base classifiers. This approach respects the natural ordering of tenure classes (short, medium, long) and penalizes distant misclassifications more heavily than adjacent errors. The ordinal model, optimized using Quadratic Weighted Kappa (QWK), achieves a QWK of 0.754, mean absolute error of 0.307, and 98.4\% adjacent accuracy on held-out test data---outperforming a standard multiclass approach on all ordinal metrics. Analysis of feature importance reveals that third-down conversion efficiency during prior head coaching roles is the strongest predictor of future tenure, followed by years of NFL position coaching experience and hiring team defensive performance. These findings provide actionable insights for NFL teams evaluating head coaching candidates.}

\keywords{NFL, head coach, tenure prediction, sports analytics, XGBoost, machine learning}

\maketitle

\section{Introduction}

What if the Denver Broncos could have avoided hiring Nathaniel Hackett? Or if the Raiders had not hired Josh McDaniels? Certainly, these teams would be in a different position today if they had hired different candidates. But more broadly, what if NFL teams could predict which head coach candidates would be successful? That is the aim of this project.

Specifically, this project attempts to predict the tenure classification of head coach hires using statistics available at the time the hire was made. This project also provides predictions for recently hired coaches.

\section{Literature Review}

At the time of writing, there have been no journal publications that attempt to predict the success of NFL coaching hires through statistical learning techniques. Currently, the NFL is only just beginning to implement artificial intelligence (AI) in play calling prediction \citep{datarobot2020}.

There are few papers that examine the impact of individual features on NFL head coaching success. \citet{roach2016} used a linear regression with seven features to attempt to predict the number of wins of head coaches in their first three years to understand if prior NFL head coaching experience impacts success in position. This paper found that previous head coaching experience had a negative impact on the success of new head coaches. Despite this finding, the model supported an adjusted $R^2$ of only 0.336. This low value, the lack of regularization, and the small number of features decreases confidence in the study's findings.

\citet{mielke2007} reviews research in sports economics and suggests that hiring decisions made solely on playing success are unlikely to be optimal given financial (resource) inequality among sports franchises.

\section{Methods}

Using statistics available at the time of hiring, this project attempts to predict the tenure classification of NFL head coach hires using XGBoost models \citep{chen2016}. Raw data was collected by scraping pro-football-reference.com. All data processing, model implementation, and analysis were performed using Python with scikit-learn \citep{pedregosa2011}.

\subsection{Predicting Coach Tenure Classification}

The tenure of a coach hire is defined as the number of years the hired coach remains in the same position before being fired, leaving for another role, or retiring. Equation~\eqref{eq:tenure_class} shows the mapping between the coach tenure $t$ (in years) and the three coach tenure classification labels $C(t)$.

\begin{equation}
C(t) = \begin{cases}
0 & \text{if } t \leq 2 \\
1 & \text{if } 2 < t \leq 4 \\
2 & \text{if } t > 4
\end{cases}
\label{eq:tenure_class}
\end{equation}

This project groups coach tenures into classes for classification rather than predicting the number of years with a regression as there is little apparent difference between coaches with similar tenures. For example, a coach that remains in position for 15 years is not 50\% more successful than a coach who remains in position for 10 years. These different coach classifications are intended to indicate different levels of coaching success based on the number of years they maintain their position.

Importantly, these tenure classes exhibit a natural ordering (Class 0 $<$ Class 1 $<$ Class 2), making ordinal classification more appropriate than standard multiclass methods. Standard multiclass approaches treat all misclassifications equally, but in this domain, predicting Class 0 for a true Class 2 coach is a more severe error than predicting Class 1.

\subsubsection{Frank-Hall Ordinal Classification}

This project implements ordinal classification using the Frank-Hall binary decomposition method \citep{frank2001}. For $K$ ordinal classes, this approach trains $K-1$ binary classifiers, each predicting the probability that an instance exceeds a given threshold. For our 3-class problem:
\begin{itemize}
    \item \textbf{Classifier 1}: $P(Y > 0)$ --- distinguishes Class 0 from Classes 1 and 2
    \item \textbf{Classifier 2}: $P(Y > 1)$ --- distinguishes Classes 0 and 1 from Class 2
\end{itemize}

Class probabilities are then derived from these cumulative probabilities:
\begin{align}
P(Y = 0) &= 1 - P(Y > 0) \\
P(Y = 1) &= P(Y > 0) - P(Y > 1) \\
P(Y = 2) &= P(Y > 1)
\end{align}

The Frank-Hall method offers several advantages: it works with any base classifier (preserving XGBoost's strengths), produces interpretable probability distributions, and naturally penalizes distant misclassifications.

\subsubsection{Evaluation Metrics}

To evaluate ordinal classification performance, this project uses metrics that account for class ordering:
\begin{itemize}
    \item \textbf{Mean Absolute Error (MAE)}: Average absolute distance between predicted and true classes
    \item \textbf{Quadratic Weighted Kappa (QWK)}: Agreement measure that penalizes distant errors more heavily than adjacent errors
    \item \textbf{Adjacent Accuracy}: Proportion of predictions within one class of the true label
    \item \textbf{AUROC}: Macro-averaged one-versus-rest area under the ROC curve
\end{itemize}

\subsubsection{Cross-Validation Strategy}

To prevent data leakage, this project implements coach-level stratified cross-validation. Since individual coaches may appear multiple times in the dataset (e.g., Bill Belichick was hired as head coach in both 1991 and 2000), all instances for a given coach are kept together in either the training or test set. This ensures the model cannot learn from a coach's prior hiring outcomes when predicting their tenure in a different role.

\subsection{Data Description}

This project utilizes 150 features, two description labels, and the two model outputs for each head coaching hire. Appendix~\ref{app:features} shows the set of 150 features used. Abbreviations included in feature descriptions include offensive coordinator (OC), defensive coordinator (DC), and head coach (HC).

Features 1--140 are characteristics of head coaches at time of hiring, while features 141--150 are characteristics of the hiring team. Features 9--140 and 141--150 reference average normalized metrics, utilizing a traditional z-score distance from league average. This normalization allows coaches across eras to be compared, as team performance is measured relative to other teams in the same year.

This project utilizes SVD Matrix Imputation to impute missing values. Figure~\ref{fig:correlation} shows the correlation matrix among all 150 features post imputation. There is some correlation within the sections of features associated with experience as an OC, DC, or HC. Appendix~\ref{app:distributions} shows the distribution of coach tenure classifications across all hiring instances in the history of the NFL.

\begin{figure}[!ht]
\centering
\includegraphics[width=0.9\textwidth]{figures/correlation_matrix.png}
\caption{Feature correlation matrix showing relationships among 150 features grouped by category: Core Experience (1--8), OC Stats (9--41), DC Stats (42--74), HC Stats (75--107), HC Opponent Stats (108--140), and Hiring Team Context (141--150).}
\label{fig:correlation}
\end{figure}

\section{Results}

The dataset contains 635 coaching hire instances with known tenure outcomes, featuring 150 engineered features per instance. The class distribution is imbalanced: Class 0 (1--2 years) comprises 49.0\% of instances, Class 1 (3--4 years) comprises 26.8\%, and Class 2 (5+ years) comprises 24.3\%.

Data was split into training (508 instances) and test (127 instances) sets using coach-level stratified sampling to prevent data leakage. Hyperparameters were tuned via 5-fold coach-level cross-validation on the training set (final values shown in Appendix~\ref{app:hyperparameters}). All reported metrics are evaluated on the held-out test set.

\subsection{Predicting Coach Tenure Classification}

\subsubsection{Ordinal Classification Model Performance}

Table~\ref{tab:tenure_results} shows the performance metrics for the ordinal XGBoost classifier on the held-out test set (127 instances). The model achieves strong performance across all ordinal metrics.

\begin{table}[!ht]
\caption{Coach tenure classification prediction results (Ordinal Model)}
\label{tab:tenure_results}
\small
\begin{tabular}{lc}
\toprule
Metric & Test Set Performance \\
\midrule
Mean Absolute Error (MAE) & 0.307 \\
Quadratic Weighted Kappa (QWK) & 0.754 \\
Adjacent Accuracy ($\pm$1 class) & 98.4\% \\
Exact Accuracy & 72.4\% \\
Macro F1 Score & 0.695 \\
AUROC (macro OVR) & 0.881 \\
\midrule
Human Baseline F1$^1$ & 0.130 \\
\midrule
Model Improvement$^2$ & 5.3$\times$ \\
\bottomrule
\multicolumn{2}{l}{\footnotesize $^1$Assuming all GMs believe their selected HC is Class 2} \\
\multicolumn{2}{l}{\footnotesize $^2$Model F1 vs. human baseline} \\
\end{tabular}
\end{table}

The ordinal model achieves a quadratic weighted kappa of 0.754, indicating substantial agreement between predictions and true labels while accounting for ordinal distance. The model achieves 98.4\% adjacent accuracy, meaning nearly all predictions are within one class of the true label---a critical property for ordinal classification.

\subsubsection{Comparison with Standard Multiclass Classification}

To validate the ordinal approach, we compare against a standard multiclass XGBoost classifier trained with the same hyperparameters. Table~\ref{tab:model_comparison} shows that the ordinal model outperforms multiclass on most metrics, particularly those that account for class ordering.

\begin{table}[!ht]
\caption{Ordinal vs.\ Multiclass model comparison on held-out test set}
\label{tab:model_comparison}
\small
\begin{tabular}{lccc}
\toprule
Metric & Ordinal & Multiclass & Better \\
\midrule
MAE & \textbf{0.307} & 0.402 & Ordinal \\
QWK & \textbf{0.754} & 0.672 & Ordinal \\
Adjacent Accuracy & \textbf{98.4\%} & 96.9\% & Ordinal \\
Exact Accuracy & \textbf{72.4\%} & 63.0\% & Ordinal \\
Macro F1 & \textbf{0.695} & 0.589 & Ordinal \\
AUROC & \textbf{0.881} & 0.836 & Ordinal \\
Class 1 F1 & \textbf{0.581} & 0.358 & Ordinal (+62.3\%) \\
\bottomrule
\end{tabular}
\end{table}

The ordinal model shows consistent improvement across all metrics, with the most notable improvement in Class 1 (middle class) F1 score. The middle class is typically most difficult to predict because it can be confused with both Class 0 and Class 2; the ordinal model's 62.3\% improvement (0.581 vs.\ 0.358) demonstrates that the Frank-Hall decomposition, combined with QWK-based hyperparameter optimization, substantially helps distinguish the intermediate tenure class.

Figure~\ref{fig:tenure_predictions} shows the sorted validation set with corresponding marks for the ground truth values and the predicted values.

\begin{figure}[!ht]
\centering
\includegraphics[width=0.98\textwidth]{figures/tenure_predictions.png}
\caption{Ordinal model predictions versus ground truth on the held-out test set. Top row shows true class; bottom row shows predicted class. Hatched bars indicate misclassifications. Instances are sorted by true class, then by predicted class within each true class.}
\label{fig:tenure_predictions}
\end{figure}

Figure~\ref{fig:tenure_importance} shows the feature weight distributions resulting from the best models found within the outer ten-fold cross-validation. These importance values for these features do not infer a monotonic relationship between feature value and predicted value. Rather, these importance values result from feature prevalence in the model's weak estimators. A feature with higher importance is present in more estimators than a feature with low importance.

\begin{figure}[!ht]
\centering
\includegraphics[width=0.9\textwidth]{figures/tenure_feature_importance.png}
\caption{Top 20 feature importances for the ordinal classifier, colored by feature category.}
\label{fig:tenure_importance}
\end{figure}

Figure~\ref{fig:category_importance} shows the total and average feature importance aggregated by category. The Core Experience category (8 features) has the highest average importance per feature, indicating that fundamental coaching experience metrics are individually more predictive than any single performance statistic. However, the HC Stats category (66 features capturing both team and opponent performance during prior head coaching roles) contributes the highest total importance, reflecting the large number of features and the predictive value of prior head coaching performance.

\begin{figure}[!ht]
\centering
\includegraphics[width=0.48\textwidth]{figures/category_importance_total.png}
\includegraphics[width=0.48\textwidth]{figures/category_importance_avg.png}
\caption{Feature importance aggregated by category. Left: Total importance (sum across all features in category). Right: Average importance (mean per feature). Core Experience features have highest average importance, while HC Stats (including opponent statistics) contribute highest total importance.}
\label{fig:category_importance}
\end{figure}

The features with the highest average importance are shown in Table~\ref{tab:tenure_feature_importance}. The most important feature is third-down conversion percentage during prior head coaching roles, highlighting that in-game execution efficiency is a strong predictor of future tenure. Multiple features related to both offensive and defensive performance during prior head coaching roles appear in the top 10, suggesting that past head coaching performance is a strong predictor of future performance. Interestingly, hiring team context (specifically yards allowed) also ranks highly, indicating that the quality of the team inherited plays a role in tenure outcomes.

\begin{table}[!ht]
\caption{Ordinal classifier feature importance for top 10 most important features}
\label{tab:tenure_feature_importance}
\small
\begin{tabular}{clc}
\toprule
Rank & Feature Description & Importance \\
\midrule
1 & During years as NFL HC, team's avg 3rd down conv. \% & 0.0341 \\
2 & Number of years' experience as NFL position coach & 0.0276 \\
3 & During years as NFL OC, team's avg turnovers & 0.0274 \\
4 & Hiring team's avg yards allowed in previous two years & 0.0269 \\
5 & During years as NFL HC, opp. team's avg rushing TDs & 0.0241 \\
6 & During years as NFL OC, team's avg points per drive & 0.0234 \\
7 & During years as NFL HC, team's avg penalty 1st downs & 0.0213 \\
8 & During years as NFL HC, team's avg passing TDs & 0.0208 \\
9 & During years as NFL DC, opp. team's avg 3rd down conv. \% & 0.0195 \\
10 & During years as NFL HC, opp. team's avg 3rd down conv. \% & 0.0193 \\
\bottomrule
\end{tabular}
\end{table}

\subsubsection{Predicting the Tenure of Recent Head Coach Hires}

Table~\ref{tab:tenure_predictions} shows the ordinal model's predictions for coach tenure for the 21 head coaches hired in the last four years. This table also shows the probabilities associated with each class prediction; these probabilities sum to 1, and the class with the greatest probability is the final predicted class. As a reminder, Class 0 represents coaches who remain a head coach for 1--2 years, Class 1 represents coaches who remain a head coach for 3--4 years, and Class 2 represents coaches who remain a head coach for 5+ years.

\textbf{Data Leakage Prevention:} Seven coaches in the prediction set (marked with *) have prior head coaching stints in the training data. To prevent data leakage---where the model could ``know'' about their past tenure outcomes---these coaches are predicted using a model retrained with their prior instances excluded. This ensures predictions are based solely on their characteristics at time of hire, not on the model having seen their past outcomes.

\begin{table}[!ht]
\caption{Ordinal classifier coach tenure predictions for 21 recent head coach hires}
\label{tab:tenure_predictions}
\small
\begin{tabular}{lccccc}
\toprule
Coach Name & Year & Pred. & P(C0) & P(C1) & P(C2) \\
\midrule
Aaron Glenn & 2025 & 0 & 56.4\% & 42.1\% & 1.5\% \\
Ben Johnson & 2025 & 0 & 68.8\% & 30.8\% & 0.4\% \\
Brian Callahan & 2024 & 0 & 68.5\% & 28.9\% & 2.6\% \\
Brian Daboll & 2022 & 0 & 77.1\% & 22.6\% & 0.3\% \\
Brian Schottenheimer & 2025 & 1 & 37.4\% & 62.0\% & 0.6\% \\
Dan Quinn* & 2024 & 1 & 6.7\% & 91.4\% & 1.9\% \\
Dave Canales & 2024 & 0 & 55.9\% & 43.1\% & 1.0\% \\
DeMeco Ryans & 2023 & 1 & 4.6\% & 93.6\% & 1.7\% \\
Jim Harbaugh* & 2024 & 0 & 69.6\% & 0.0\% & 30.4\% \\
Jonathan Gannon & 2023 & 1 & 48.0\% & 49.4\% & 2.6\% \\
Kellen Moore & 2025 & 0 & 73.1\% & 26.5\% & 0.4\% \\
Kevin O'Connell & 2022 & 1 & 24.0\% & 68.5\% & 7.5\% \\
Liam Coen & 2025 & 0 & 51.4\% & 44.8\% & 3.8\% \\
Mike Macdonald & 2024 & 1 & 21.9\% & 69.0\% & 9.1\% \\
Mike McDaniel & 2022 & 0 & 75.0\% & 24.1\% & 0.9\% \\
Mike Vrabel* & 2025 & 2 & 0.6\% & 40.7\% & 58.7\% \\
Pete Carroll* & 2025 & 1 & 8.8\% & 64.3\% & 26.9\% \\
Raheem Morris* & 2024 & 0 & 95.7\% & 2.3\% & 1.9\% \\
Sean Payton* & 2023 & 1 & 10.2\% & 89.2\% & 0.6\% \\
Shane Steichen & 2023 & 0 & 62.8\% & 36.9\% & 0.4\% \\
Todd Bowles* & 2022 & 0 & 86.8\% & 12.3\% & 0.8\% \\
\bottomrule
\end{tabular}
\vspace{0.5em}
\newline
{\footnotesize *Predicted with model retrained to exclude coach's prior HC data}
\end{table}

After applying the data leakage fix, the ordinal model predicts only 1 of the 21 recent hires to achieve Class 2 (5+ years): Mike Vrabel (58.7\% confidence). Notably, Pete Carroll---who would have been predicted Class 2 without the leakage fix due to the model having learned from his successful 14-year Seattle tenure---is instead predicted Class 1 (64.3\%) when his prior outcomes are excluded from training. This demonstrates the importance of preventing data leakage for fair predictions. The model shows highest confidence (95.7\%) that Raheem Morris will have a short tenure, while DeMeco Ryans and Dan Quinn receive strong Class 1 predictions (93.6\% and 91.4\%, respectively), suggesting moderate expected tenure with high confidence.

\section{Conclusion}

The ordinal classification approach using the Frank-Hall binary decomposition method, optimized with Quadratic Weighted Kappa (QWK), demonstrates strong predictive performance for NFL head coach tenure. The model achieves a QWK of 0.754, AUROC of 0.881, and 98.4\% adjacent accuracy on held-out test data, indicating that predictions are both accurate and, when incorrect, typically only off by one class. Compared to a standard multiclass approach, the ordinal model shows improvements across all metrics, with a notable 62.3\% improvement in the challenging middle-class (3--4 year tenure) F1 score (0.581 vs.\ 0.358).

Feature importance analysis reveals that third-down conversion efficiency during prior head coaching roles is the strongest predictor of future tenure, suggesting that in-game execution and coaching decisions under pressure are more predictive than raw offensive or defensive statistics. The number of years of experience as an NFL position coach also ranks highly, indicating that foundational coaching experience contributes to tenure longevity. Interestingly, hiring team context (specifically yards allowed in previous seasons) also ranks among the top predictors, suggesting that the quality of the team inherited plays a meaningful role in tenure outcomes.

This research demonstrates that head coach characteristics---particularly prior head coaching performance and position coaching experience---can meaningfully predict tenure classification. For NFL teams, this suggests that hiring decisions should focus on candidates with strong prior head coaching performance (particularly third-down efficiency) and extensive position coaching experience. The model's predictions for recent hires, including high confidence that Mike Vrabel will achieve long tenure and that Raheem Morris faces a short tenure, provide actionable insights for evaluating current coaching situations.

\begin{thebibliography}{99}

\bibitem[Chen and Guestrin(2016)]{chen2016}
Chen, T. and Guestrin, C. (2016). XGBoost: a scalable tree boosting system. In: \emph{KDD '16: Proceedings of the 22nd ACM SIGKDD International Conference on Knowledge Discovery and Data Mining}, pp. 785--794.

\bibitem[DataRobot(2020)]{datarobot2020}
DataRobot (2020). Using machine learning to peek inside the minds of NFL coaches. Available at: https://www.datarobot.com/blog/using-machine-learning-to-peek-inside-the-minds-of-nfl-coaches/

\bibitem[Frank and Hall(2001)]{frank2001}
Frank, E. and Hall, M. (2001). A simple approach to ordinal classification. In: \emph{Machine Learning: ECML 2001}, Lecture Notes in Computer Science, vol. 2167. Springer, Berlin, Heidelberg, pp. 145--156.

\bibitem[Mielke(2007)]{mielke2007}
Mielke, D. (2007). Coaching experience, playing experience, and coaching tenure: a commentary. \emph{International Journal of Sports Science \& Coaching}, \textbf{2}(2), pp. 117--118.

\bibitem[Pedregosa et al.(2011)]{pedregosa2011}
Pedregosa, F. et al. (2011). Scikit-learn: machine learning in Python. \emph{Journal of Machine Learning Research}, \textbf{12}, pp. 2825--2830.

\bibitem[Roach(2016)]{roach2016}
Roach, M. (2016). Does prior NFL head coaching experience improve team performance? \emph{Journal of Sport Management}, \textbf{30}(3), pp. 298--311.

\end{thebibliography}

\clearpage

\appendix

\section{Feature Descriptions}
\label{app:features}

\begin{table}[!ht]
\caption{Feature descriptions (Features 1--41)}
\label{tab:features_1}
\scriptsize
\begin{tabular}{cl}
\toprule
No. & Feature Description \\
\midrule
1 & Age at hiring \\
2 & Number of times previously hired as head coach \\
3 & Number of years' experience as college position coach \\
4 & Number of years' experience as college coordinator \\
5 & Number of years' experience as college head coach \\
6 & Number of years' experience as NFL position coach \\
7 & Number of years' experience as NFL coordinator \\
8 & Number of years' experience as NFL head coach \\
9 & During years as NFL OC, team's average points scored \\
10 & During years as NFL OC, team's average yards \\
11 & During years as NFL OC, team's average yards/play \\
12 & During years as NFL OC, team's average turnovers \\
13 & During years as NFL OC, team's average 1st downs \\
14 & During years as NFL OC, team's average passing completions \\
15 & During years as NFL OC, team's average passing attempts \\
16 & During years as NFL OC, team's average passing yards \\
17 & During years as NFL OC, team's average passing touchdowns \\
18 & During years as NFL OC, team's average passing interceptions \\
19 & During years as NFL OC, team's average NY/A \\
20 & During years as NFL OC, team's average passing first downs \\
21 & During years as NFL OC, team's average rushing attempts \\
22 & During years as NFL OC, team's average rushing yards \\
23 & During years as NFL OC, team's average rushing touchdowns \\
24 & During years as NFL OC, team's average rush yards per play \\
25 & During years as NFL OC, team's average rushing 1st downs \\
26 & During years as NFL OC, team's average number of penalties \\
27 & During years as NFL OC, team's average penalty yards \\
28 & During years as NFL OC, team's average penalty 1st downs \\
29 & During years as NFL OC, team's average number of drives \\
30 & During years as NFL OC, team's average scoring percentage \\
31 & During years as NFL OC, team's average turnover percentage \\
32 & During years as NFL OC, team's average drive duration \\
33 & During years as NFL OC, team's average plays per drive \\
34 & During years as NFL OC, team's average yards per drive \\
35 & During years as NFL OC, team's average points per drive \\
36 & During years as NFL OC, team's average number of 3rd down attempts \\
37 & During years as NFL OC, team's average third down conversion percentage \\
38 & During years as NFL OC, team's average number of 4th down attempts \\
39 & During years as NFL OC, team's average 4th down conversion percentage \\
40 & During years as NFL OC, team's average red zone attempts \\
41 & During years as NFL OC, team's average red zone percentage \\
\bottomrule
\end{tabular}
\end{table}

\begin{table}[!ht]
\caption{Feature descriptions (Features 42--74)}
\label{tab:features_2}
\scriptsize
\begin{tabular}{cl}
\toprule
No. & Feature Description \\
\midrule
42 & During years as NFL DC, opponent team's average points scored \\
43 & During years as NFL DC, opponent team's average yards \\
44 & During years as NFL DC, opponent team's average yards/play \\
45 & During years as NFL DC, opponent team's average turnovers \\
46 & During years as NFL DC, opponent team's average 1st downs \\
47 & During years as NFL DC, opponent team's average passing completions \\
48 & During years as NFL DC, opponent team's average passing attempts \\
49 & During years as NFL DC, opponent team's average passing yards \\
50 & During years as NFL DC, opponent team's average passing touchdowns \\
51 & During years as NFL DC, opponent team's average passing interceptions \\
52 & During years as NFL DC, opponent team's average NY/A \\
53 & During years as NFL DC, opponent team's average passing first downs \\
54 & During years as NFL DC, opponent team's average rushing attempts \\
55 & During years as NFL DC, opponent team's average rushing yards \\
56 & During years as NFL DC, opponent team's average rushing touchdowns \\
57 & During years as NFL DC, opponent team's average rush yards per play \\
58 & During years as NFL DC, opponent team's average rushing 1st downs \\
59 & During years as NFL DC, opponent team's average number of penalties \\
60 & During years as NFL DC, opponent team's average penalty yards \\
61 & During years as NFL DC, opponent team's average penalty 1st downs \\
62 & During years as NFL DC, opponent team's average number of drives \\
63 & During years as NFL DC, opponent team's average scoring percentage \\
64 & During years as NFL DC, opponent team's average turnover percentage \\
65 & During years as NFL DC, opponent team's average drive duration \\
66 & During years as NFL DC, opponent team's average plays per drive \\
67 & During years as NFL DC, opponent team's average yards per drive \\
68 & During years as NFL DC, opponent team's average points per drive \\
69 & During years as NFL DC, opponent team's average number of 3rd down attempts \\
70 & During years as NFL DC, opponent team's average third down conversion pct. \\
71 & During years as NFL DC, opponent team's average number of 4th down attempts \\
72 & During years as NFL DC, opponent team's average 4th down conversion pct. \\
73 & During years as NFL DC, opponent team's average red zone attempts \\
74 & During years as NFL DC, opponent team's average red zone percentage \\
\bottomrule
\end{tabular}
\end{table}

\begin{table}[!ht]
\caption{Feature descriptions (Features 75--107)}
\label{tab:features_3}
\scriptsize
\begin{tabular}{cl}
\toprule
No. & Feature Description \\
\midrule
75 & During years as NFL HC, team's average points scored \\
76 & During years as NFL HC, team's average yards \\
77 & During years as NFL HC, team's average yards/play \\
78 & During years as NFL HC, team's average turnovers \\
79 & During years as NFL HC, team's average 1st downs \\
80 & During years as NFL HC, team's average passing completions \\
81 & During years as NFL HC, team's average passing attempts \\
82 & During years as NFL HC, team's average passing yards \\
83 & During years as NFL HC, team's average passing touchdowns \\
84 & During years as NFL HC, team's average passing interceptions \\
85 & During years as NFL HC, team's average NY/A \\
86 & During years as NFL HC, team's average passing first downs \\
87 & During years as NFL HC, team's average rushing attempts \\
88 & During years as NFL HC, team's average rushing yards \\
89 & During years as NFL HC, team's average rushing touchdowns \\
90 & During years as NFL HC, team's average rush yards per play \\
91 & During years as NFL HC, team's average rushing 1st downs \\
92 & During years as NFL HC, team's average number of penalties \\
93 & During years as NFL HC, team's average penalty yards \\
94 & During years as NFL HC, team's average penalty 1st downs \\
95 & During years as NFL HC, team's average number of drives \\
96 & During years as NFL HC, team's average scoring percentage \\
97 & During years as NFL HC, team's average turnover percentage \\
98 & During years as NFL HC, team's average drive duration \\
99 & During years as NFL HC, team's average plays per drive \\
100 & During years as NFL HC, team's average yards per drive \\
101 & During years as NFL HC, team's average points per drive \\
102 & During years as NFL HC, team's average number of 3rd down attempts \\
103 & During years as NFL HC, team's average third down conversion percentage \\
104 & During years as NFL HC, team's average number of 4th down attempts \\
105 & During years as NFL HC, team's average 4th down conversion percentage \\
106 & During years as NFL HC, team's average red zone attempts \\
107 & During years as NFL HC, team's average red zone percentage \\
\bottomrule
\end{tabular}
\end{table}

\begin{table}[!ht]
\caption{Feature descriptions (Features 108--150)}
\label{tab:features_4}
\scriptsize
\begin{tabular}{cl}
\toprule
No. & Feature Description \\
\midrule
108 & During years as NFL HC, opponent team's average points scored \\
109 & During years as NFL HC, opponent team's average yards \\
110 & During years as NFL HC, opponent team's average yards/play \\
111 & During years as NFL HC, opponent team's average turnovers \\
112 & During years as NFL HC, opponent team's average 1st downs \\
113 & During years as NFL HC, opponent team's average passing completions \\
114 & During years as NFL HC, opponent team's average passing attempts \\
115 & During years as NFL HC, opponent team's average passing yards \\
116 & During years as NFL HC, opponent team's average passing touchdowns \\
117 & During years as NFL HC, opponent team's average passing interceptions \\
118 & During years as NFL HC, opponent team's average NY/A \\
119 & During years as NFL HC, opponent team's average passing first downs \\
120 & During years as NFL HC, opponent team's average rushing attempts \\
121 & During years as NFL HC, opponent team's average rushing yards \\
122 & During years as NFL HC, opponent team's average rushing touchdowns \\
123 & During years as NFL HC, opponent team's average rush yards per play \\
124 & During years as NFL HC, opponent team's average rushing 1st downs \\
125 & During years as NFL HC, opponent team's average number of penalties \\
126 & During years as NFL HC, opponent team's average penalty yards \\
127 & During years as NFL HC, opponent team's average penalty 1st downs \\
128 & During years as NFL HC, opponent team's average number of drives \\
129 & During years as NFL HC, opponent team's average scoring percentage \\
130 & During years as NFL HC, opponent team's average turnover percentage \\
131 & During years as NFL HC, opponent team's average drive duration \\
132 & During years as NFL HC, opponent team's average plays per drive \\
133 & During years as NFL HC, opponent team's average yards per drive \\
134 & During years as NFL HC, opponent team's average points per drive \\
135 & During years as NFL HC, opponent team's average number of 3rd down attempts \\
136 & During years as NFL HC, opponent team's average third down conversion pct. \\
137 & During years as NFL HC, opponent team's average number of 4th down attempts \\
138 & During years as NFL HC, opponent team's average 4th down conversion pct. \\
139 & During years as NFL HC, opponent team's average red zone attempts \\
140 & During years as NFL HC, opponent team's average red zone percentage \\
141 & Hiring team's average winning percent in previous two years \\
142 & Hiring team's average points scored in previous two years \\
143 & Hiring team's average points allowed in previous two years \\
144 & Hiring team's average yards of offense in previous two years \\
145 & Hiring team's average yards of offense allowed in previous two years \\
146 & Hiring team's average yards / play in previous two years \\
147 & Hiring team's average yards / play allowed in previous two years \\
148 & Hiring team's average turnovers forced in previous two years \\
149 & Hiring team's average turnovers in previous two years \\
150 & Hiring team's number of playoff appearances in previous two years \\
\bottomrule
\end{tabular}
\end{table}

\clearpage

\section{Data Distributions}
\label{app:distributions}

\begin{figure}[!ht]
\centering
\includegraphics[width=0.8\textwidth]{figures/tenure_distribution.png}
\caption{Coach tenure classification frequency distribution across all 635 coaching hire instances with known tenure outcomes.}
\label{fig:tenure_dist}
\end{figure}

\clearpage

\section{Model Hyperparameters}
\label{app:hyperparameters}

\begin{table}[!ht]
\caption{Final hyperparameters for the ordinal XGBoost classifier model (QWK-optimized)}
\label{tab:classifier_hyperparams}
\begin{tabular}{lc}
\toprule
Hyperparameter & Value \\
\midrule
Classification Method & Frank-Hall Ordinal \\
Optimization Metric & Quadratic Weighted Kappa \\
Base Classifier Objective & binary:logistic \\
Number of Binary Classifiers & 2 \\
Random State & 42 \\
Number of Estimators & 200 \\
Learning Rate & 0.25 \\
Max Estimator Depth & 2 \\
Gamma & 0 \\
Lambda (L2 Regularization) & 0.1 \\
Alpha (L1 Regularization) & 0.01 \\
Subsample & 0.80 \\
Colsample by Tree & 0.90 \\
Minimum Child Weight & 3 \\
\bottomrule
\end{tabular}
\end{table}

\end{document}
